\documentclass[]{article}
\usepackage{fontspec}
\usepackage{parskip}
\usepackage{amsmath}

\renewcommand{\familydefault}{\sfdefault}

\begin{document}

%title
\begingroup  
	\raggedright
	\LARGE Foundations of Analog and Digital Electronic Circuits\\
	\LARGE Exercise and Problem Solutions\\
	\vspace{3mm} %3 mm vertical space
	\normalsize Henrik Samuelsson\\
	\normalsize henrik.samuelsson@gmail.com
\endgroup
	
\section*{About}
	This is my personal solutions to exercises and problems from the book Foundations of Analog and Digital Electronic Circuits. Very much appreciated if you mail me if you find errors in the solutions. My mail address can be found above this section.


\begin{enumerate}
	
	%Exercise 1.1
	\item Assume a 5k and a 10k resistor.
	
	\begin{enumerate}
		
		%(a)
		\item The combined resistance if connecting these resistors in series is calculated by adding the resistances.
		
		\begin{equation*}
		R = R_{5k} + R_{10k} = 5k + 10k = 15k
		\end{equation*}
		
		%(b)
		\item The combined resistance if connecting these resistors in parallel can calculated in two different ways.
		
		\begin{equation*}
				R = \dfrac{ R_{5k} R_{10k} }{ R_{5k} + R_{10k} } = \dfrac{ 5k \times 1k }{ 5k + 10k } = 3.3k
		\end{equation*}
		
		\begin{equation*}
			R = \dfrac{ 1 }{ \dfrac{ 1 }{ R_{5k} } + \dfrac { 1 } { R_{10k} } }
			= \dfrac{ 1 }{ \dfrac{ 1 }{ 5k } + \dfrac { 1 } { 10k } } 
			= \dfrac {1k}{0.2 + 0.1}= 3.3k
		\end{equation*}
		
	\end{enumerate}
		
	%Exercise 1.2
	\item We shall calculate the power dissipated when connecting a 12V car battery across a 1 ohm resistor.
		
	Ohm's law gives the current that will flow through the resistor.
		
	\begin{equation*}
		I =\dfrac{ V }{ R } = \dfrac{ 12 }{ 1 } = 12A
	\end{equation*}
	
	The dissipated power is calculated by multiplying the voltage across the resistor with the current that flows through it.
	\begin{equation*}
		P = VI = 12 \times 12 = 144W
	\end{equation*}
		
\end{enumerate}
	
\end{document}