\documentclass[]{article}
\usepackage{parskip}
\usepackage{amsmath}
\usepackage{enumitem}

\begin{document}

%title
\begingroup  
	\raggedright
	\Large Foundations of Analog and Digital Electronic Circuits\\
	\Large Exercise and Problems Solutions\\
	\vspace{3mm} %3 mm vertical space
	\normalsize Henrik Samuelsson\\
	\normalsize henrik.samuelsson@gmail.com
\endgroup
	
\section*{About}
	This is my personal solutions to exercises from the book Foundations of Analog and Digital Electronic Circuits. Very much appreciated if you mail me if you find errors in the solutions. My mail address can be found above this section.

\section*{Chapter 1}

\subsection*{Exercise 1.1}
	
	Then the power dissipated in a purely resistive load fed from an AC supply is given as
	
	\begin{equation*}
		P = \dfrac{ V^{2}_{rms} }{ R }
	\end{equation*}
		
	Rearrangement and insertion of known values gives the answer
	
	\begin{equation*}
		R = \dfrac{ V^{2}_{rms} }{ P } = \dfrac{ 120^{2} }{ 1200 } = 12 \Omega
	\end{equation*}	

\subsection*{Exercise 1.2}

\begin{enumerate}[label=(\alph*)]
	\item The energy stored in a fully charged 50 A-hour 12-V battery is

	\begin{equation*}
		12 \cdot 50 \cdot 3600 = 2.16 MJ
	\end{equation*}

	\item The energy of water in a dam can be calculated by the following formula
		
	\begin{equation*}
		E =  m \cdot g \cdot h
	\end{equation*}

	We want to calculate the mass m, rearranging of the above formula gives
	
	\begin{equation*}
		m = \dfrac{E}{g \cdot h} = \dfrac{2.16 \cdot 10^6}{9.82 \cdot 30} = 7331 kg
	\end{equation*}
	
	This means that it takes 7 331 litres of water from a 30 meters high dam to charge the battery. 
\end{enumerate}
	
\end{document}
