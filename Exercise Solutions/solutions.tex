\documentclass[]{article}
\usepackage{parskip}
\usepackage{amsmath}
\usepackage{enumitem}

\newcommand*\average[1]{\bar{#1}}

\begin{document}

%title
\begingroup  
	\raggedright
	\Large Foundations of Analog and Digital Electronic Circuits\\
	\Large Exercise and Problems Solutions\\
	\vspace{3mm} %3 mm vertical space
	\normalsize Henrik Samuelsson\\
	\normalsize henrik.samuelsson@gmail.com
\endgroup
	
\section*{About}
	This is my personal solutions to exercises and problems from the book Foundations of Analog and Digital Electronic Circuits. A copy of the book is needed to fully understand the solutions since I will not include the full exercise instructions here. Exact version of the book is from 2005 and it has the ISBN number 978-1-55860-735-4. 
	
	Very much appreciated if you contact me when you find errors in the solutions. My e-mail address can be found above this section.

\section*{Chapter 1}

\subsection*{Exercise 1.1}
	
	Then the power dissipated in a purely resistive load fed from an AC supply is given as
	
	\begin{equation*}
		P = \dfrac{ V^{2}_{rms} }{ R }
	\end{equation*}
		
	Rearrangement and insertion of known values gives the answer
	
	\begin{equation*}
		R = \dfrac{ V^{2}_{rms} }{ P } = \dfrac{ 120^{2} }{ 1200 } = 12 \Omega
	\end{equation*}	

\subsection*{Exercise 1.2}

\begin{enumerate}[label=(\alph*)]
	\item The energy stored in a fully charged 50 A-hour 12-V battery is

	\begin{equation*}
		12 \cdot 50 \cdot 3600 = 2.16 MJ
	\end{equation*}

	\item The energy of water in a dam can be calculated by the following formula
		
	\begin{equation*}
		E =  m \cdot g \cdot h
	\end{equation*}

	We want to calculate the mass m, rearranging of the above formula gives
	
	\begin{equation*}
		m = \dfrac{E}{g \cdot h} = \dfrac{2.16 \cdot 10^6}{9.82 \cdot 30} = 7331 kg
	\end{equation*}
	
	This means that it takes 7 331 litres of water from a 30 meters high dam to charge the battery. 
\end{enumerate}

\subsection*{Exercise 1.3}

The power dissipated in the resistor is given by

\begin{equation*}
	p =  \dfrac{ V _{DC} ^{2} }{ R }
\end{equation*}

\subsection*{Exercise 1.4}

\begin{enumerate}[label=(\alph*)]
	\item The average power dissipated in R is

	\begin{equation*}
		\average{p} =  { V _{rms} ^{2} }\dfrac{ 1 }{ R } = \left( \dfrac{ V _{AC} }{ \sqrt{2} }\right) ^2 \cdot \dfrac{ 1 }{ R }
	\end{equation*}

	\item
	\begin{equation*}
		V _{DC} = \dfrac{ V _{AC} }{ \sqrt{2} }
	\end{equation*}	
 
\end{enumerate}
	
\end{document}
